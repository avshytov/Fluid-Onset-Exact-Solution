\documentclass[preprint,aps,eqsecnum]{revtex4-1}
\usepackage{bm,amssymb,amsfonts,amsmath,graphicx,hyperref,subfig}
\newcommand{\fplus}[1]{{#1}^{+}}
\newcommand{\fminus}[1]{{#1}^{-}}
\newcommand{\fplusminus}[1]{{#1}^{\pm}}
\renewcommand{\Re}{\mathop{\mathrm{Re}}\nolimits}
\renewcommand{\Im}{\mathop{\mathrm{Im}}\nolimits}
\newcommand{\sgn}{\mathop{\mathrm{sgn}}\nolimits}
\newcommand{\dct}[1]{{#1}_\mathrm{direct}}
\captionsetup[subfigure]{justification=justified,singlelinecheck=false}
\captionsetup{justification=centerlast,singlelinecheck=false}
%\newcommand{\rhoplus}{\rho^{(\plus)}}
%\newcommand{\rhominus}{\rho^{(\minus)}}
%\newcommand{\jplus}{j^{(\plus)}}
%\newcommand{\jminus}{j^{(\minus)}}
\begin{document}
\title{Manifestations of the odd viscosity at the onset of fluidity}
\author{people}
\affiliation{places}
\maketitle

We write the kinetic equation in the form
\begin{align}
  {\bm v} \cdot\nabla f - \omega_c \frac{\partial f}{\partial \alpha}
  = J(\alpha) + I[f]\ , 
\end{align}
with the collision integral of the form
\begin{align}
  I[f] = -\gamma f +  \gamma \rho + 2 \gamma' {\bm j}\cdot {\bm v}\ .
\end{align}
Here~${\bm v}$ is the unit vector in the direction of particle's
velocity, $|{\bm v}| = 1$, and $\alpha$ is the propagation
direction: ${\bm v} = (\cos\alpha, \sin\alpha)$.
The quantity~$\gamma$ is the total collision rate, while
$\gamma'$ is the rate of momentum-conserving collisions.
The number density~${\rho}({\bm r})$ and current density~${\bm j}$
are defined via
\begin{align}
  \rho({\bm r}) \equiv \left\langle f({\bm r}) \right\rangle \ ,
  \qquad
  {\bm j}({\bm r}) \equiv \left\langle f({\bm r}) {\bm v} \right\rangle\ , 
\end{align}
where the brackets denote angular averaging:
\begin{align}
  \left\langle \ldots \right\rangle \equiv \oint \frac{d\alpha}{2\pi} (\ldots)\ .
\end{align}
The equation is supplied with the source term~$J({\bm r}, \alpha)$
which represents both particles injected into the system and
the particles scattered by diffuse walls. The magnetic field
enters via the cyclotron frequency~$\omega_c$.

In the previous article, we explained how the equation
can be solved in the absence of the magnetic field in the Fourier
representation defined by
\begin{align}
  f({\bm r}) = \int \frac{d^2 {\bm k}}{(2\pi)^2} e^{-i{\bm k}\cdot{\bm r}}
  f_{\bm k}\ ,
\end{align}
where~${\bm k}$ is the wave vector.
(Note that the convention here is opposite to the one employed in the
solid-state physics and instead follows the one employed in the literature
on the Wiener-Hopf method.)
This was achieved by extending the dynamics into the
lower semispace~$y < 0$ so that the particles only decay there but
not rescattered. The equation then takes the form
\begin{align}
  (\gamma - i {\bm k} \cdot {\bm v}) f_{\bm k} =
  J_{\bm k}(\alpha) + \gamma \fplus{\rho}_{\bm k}
  + 2\gamma' \fplus{\bm j}_{\bm k}
  \cdot {\bm v}\ , 
\end{align}
where~$\fplus{\rho}_{\bm k}$ indicates the Fourier image of
the number density restricted to~$y > 0$, the same convention
applies to other quantities. 
Then, the kinetic equation is solved in the form
\begin{align}
  f_{\bm k} = \frac{J_{\bm k}(\alpha) + \gamma \fplus{\rho}_{\bm k}
    + 2\gamma' \fplus{{\bm j}\cdot{\bm v}}_{\bm k}}{\gamma - i {\bm k} \cdot {\bm v}}\ . 
\end{align}
It was also convenient to rewrite the current in terms of
its divergence~$D_{\bm k} = {\bm k} \cdot {\bm j}_{\bm k}$
and vorticity~$\Omega_{\bm k} = {\bm k} \times {\bm j}_{\bm k}$.
This brings the distribution function to the following form:
\begin{align}
  f_{\bm k} = \frac{2i\gamma' D_{\bm k}}{\bm k^2} +
  \frac{1}{\gamma - i {\bm k} \cdot {\bm v}}
  \left[J_{\bm k}(\alpha) + \gamma \fplus{\rho}_{\bm k}
  - \frac{2i\gamma' \gamma}{{\bm k}^2} \fplus{D}_{\bm k} 
  + \frac{2\gamma'}{\bm k^2} \fplus{\Omega}_{\bm k} {\bm k} \times {\bm v}
  \right]\ . 
\end{align}
Then, the self-consistent equations for the quantities
$\rho_{\bm k}$, $D_{\bm k}$, and $\Omega_{\bm k}$  are
derived by  weighting~$1$, ${\bm k} \cdot{\bm v}$ and ${\bm k}\times{\bm v}$
respectively with this distribution function. This yields
\begin{align}
K_\rho({\bm k}) \fplus{\rho}_{\bm k}
+ \fminus{\rho}_{\bm k} &= \rho_\mathrm{direct}({\bm k})
+ \frac{2 i \gamma' \fplus{D}_{\bm k}}{k^2} K_\rho({\bm k}) \ , \\
\fplus{D}_{\bm k} + \fminus{D}_{\bm k}
&= -i \gamma \fminus{\rho}_{\bm k} + i \left\langle J_{\bm k}(\alpha)
  \right\rangle \ , \\
K_\Omega({\bm k}) \fplus{\Omega}_{\bm k}  + \fminus{\Omega}_{\bm k}
&= \Omega_\mathrm{direct}({\bm k})\ , 
\end{align}
where the terms~$ \rho_\mathrm{direct}({\bm k})$
and~$\Omega_\mathrm{direct}({\bm k})$ represent the external
sources, while the kernels~$K_\rho({\bm k})$ and~$K_\Omega({\bm k})$
are given by
\begin{align}
  K_\rho({\bm k}) = 1 - \frac{\gamma}{\sqrt{{\bm k}^2 + \gamma^2}}\ ,
  \qquad
  K_\Omega({\bm k}) = 1 - \frac{2\gamma'}{{\bm k}^2}
  \left(\sqrt{{\bm k}^2 + \gamma^2} - \gamma\right)\ . 
\end{align}
The source terms are given by
\begin{align}
  \rho_\mathrm{direct}({\bm k})
  = \left\langle \frac{J_{\bm k}(\alpha)}{\gamma -i {\bm k}\cdot{\bm v}} \right\rangle
  \ ,
  \qquad
  \Omega_\mathrm{direct}({\bm k}) =
  \left\langle \frac{{\bm k}\times{\bm v} J_{\bm k}(\alpha)}{
       \gamma -i {\bm k}\cdot{\bm v}} \right\rangle \ . 
\end{align}
Although the singular integral equations formally decouple the density
from vorticity, they are implicitly related via the
consistency condition:
\begin{align}
  D_{\bm k}(q = i_|k|) = i \sgn k \Omega_{\bm k}(q = i |k|)\ , 
\end{align}
where~$k$ and~$q$ are the components of~${\bm k} = (k, q)$. 

Let us now develop the first-order perturbation theory in the field.
To this end, we move the magnetic field term to the right-hand
side, and replace the distribution function~$f$ by the first-order
contribution:
\begin{align}
  (\gamma - i {\bm k} \cdot {\bm v})\delta f =
  \gamma \delta \rho_{\bm k} + 2 \gamma' \delta{\bm j}_{\bm k}
  \cdot {\bm v} + \omega_c \frac{\partial f}{\partial \alpha}\ ,
\end{align}
where the symbol~$\delta$ indicates the first-order correction
to the respective quantities. One may formally treat the
term proportional to~$\omega_c$ as an extra source and evaluate
the respective contributions to~$\delta \rho_\mathrm{direct}$,
and~$\delta \Omega_\mathrm{direct}$:
\begin{align}
  \delta \rho_\mathrm{direct}({\bm k})
  = \omega_c \left\langle \frac{1}{\gamma  - i {\bm k}\cdot{\bm v}}
     \frac{\partial f_{\bm k}}{\partial\alpha}
  \right\rangle
  \ ,
  \qquad
  \delta \Omega_\mathrm{direct}({\bm k})
  = \omega_c \left\langle \frac{{\bm k}\times{\bm v}}{
          \gamma  - i {\bm k}\cdot{\bm v}}
     \frac{\partial f_{\bm k}}{\partial\alpha}
  \right\rangle  \ .
\end{align}
By recalling that the averaging is performed by integrating over
all propagation angles~$\alpha$, one may shift the derivative
via integration by parts:
\begin{align}
  \delta \rho_\mathrm{direct}({\bm k})
  = - \omega_c \left\langle f_{\bm k}(\alpha)
  \frac{\partial}{\partial\alpha} \frac{1}{\gamma
      - i {\bm k}\cdot{\bm \alpha}}
  \right\rangle\ ,
  \qquad
  \delta \Omega_\mathrm{direct}({\bm k})
  = - \omega_c \left\langle f_{\bm k}(\alpha)
  \frac{\partial}{\partial\alpha} \frac{{\bm k}\times{\bm v}}{\gamma
      - i {\bm k}\cdot{\bm \alpha}}
  \right\rangle\ .
\end{align}
Since~$\partial_\alpha({\bm k}\cdot{\bm v}) = - {\bm k}\times{\bm v}$,
and~$\partial_\alpha({\bm k}\times{\bm v}) =  {\bm k}\cdot{\bm v}$,
we rewrite the extra sources in the form
\begin{align}
  \delta \rho_\mathrm{direct}({\bm k}) &= i \omega_c
  \left\langle f_{\bm k} \frac{{\bm k}\times{\bm v}}{
  (\gamma  - i {\bm k}\cdot{\bm v})^2} \right\rangle \ ,
  \\
  \delta \Omega_\mathrm{direct}({\bm k})
  &= i \omega_c \left\langle f_{\bm k} \left(
  \frac{\gamma^2 + {\bm k}^2}{(\gamma - i {\bm k}\cdot{\bm v})^2}
  - \frac{\gamma}{\gamma - i {\bm k} \cdot{\bm v}}
  \right)\right\rangle
\end{align}
We can now use the zeroth-order solution for~$f_{\bm k}$ given above.
This results in several terms: those that involve unperturbed
values of~$\rho_{\bm k}$, $D_{\bm k}$ and~$\Omega_{\bm k}$,
and those that involve external sources. Let us begin with the
intrinsic contributions.

In performing the averages, one may use the following relations:
\begin{align}
  \left\langle
  \frac{({\bm k}\times{\bm v})^2}{(\gamma - i {\bm k}\cdot{\bm v})^3}
  \right\rangle =
  \frac{{\bm k}^2}{2\left({\bm k}^2 + \gamma^2\right)^{3/2}}\
  \ , 
  \qquad
  \left\langle                                                                
  \frac{1}{(\gamma - i {\bm k}\cdot{\bm v})^3}    
  \right\rangle = \frac{2\gamma^2 - {\bm k}^2}{
  2 \left({\bm k}^2 + \gamma^2\right)^{5/2}} \\
  \left\langle                                                                
  \frac{1}{(\gamma - i {\bm k}\cdot{\bm v})^2}    
  \right\rangle = \frac{\gamma}{
   \left({\bm k}^2 + \gamma^2\right)^{3/2}} \ . 
\end{align}
These relations can be obtained by repeated differentiation
of the integrals
\begin{align}
  \left\langle \frac{1}{\gamma - i {\bm k}\cdot{\bm v}} \right\rangle
  = \frac{1}{\sqrt{\gamma^2 + {\bm k}^2}}\ ,
  \left\langle \frac{({\bm k}\times{\bm v})^2}{
  \gamma - i {\bm k}\cdot{\bm v}} \right\rangle
  = \sqrt{\gamma^2 + {\bm k}^2} - \gamma \ .  
\end{align}


It is clear that~$\rho_{\bm k}$
and~$D_{\bm k}$ do not contribute to~$ \delta \rho_\mathrm{direct}({\bm k})$.
The contribution of the vorticity to the density is given by
\begin{align}
  \delta \rho_{\mathrm{direct},\Omega}
  = \frac{2i\gamma' \omega_c\fplus{\Omega}_{\bm k}}{{\bm k}^2} \left\langle
  \frac{({\bm k}\times{\bm v})^2}{(\gamma - i {\bm k}\cdot{\bm v})^3}
  \right\rangle
  = \frac{i\omega_c \gamma'\fplus{\Omega}_{\bm k}}{
       \left({\bm k}^2 + \gamma^2\right)^{3/2}} \ , 
\end{align}
Similarly, the contribution of the density to the vorticity is given by
\begin{align}
  \delta\Omega_{\mathrm{direct}, \rho} =
  \frac{-i \gamma \omega_c {\bm k}^2}{2\left(\gamma^2 + {\bm k}^2\right)^{3/2}}
  \left( - \frac{2i\gamma' \fplus{D}_{\bm k}}{{\bm k}^2}
     + \fplus{\rho}_{\bm k}\right)\ . 
\end{align}
(One may also incorporate the isotropic part of the source,
$\langle J_{\bm k}(\alpha) \rangle$ into the bracket on the
right-hand side.)

The contribution of the external sources,
\begin{align}
  \delta \rho_{\mathrm{direct},J}({\bm k}) &=
  i \omega_c \left\langle \frac{J_{\bm k}(\alpha) ({\bm k}\times{\bm v})}{
     (\gamma - i {\bm k}\cdot{\bm v})^3}\right\rangle\ , \\  
  \delta \Omega_{\mathrm{direct},J}({\bm k}) &=
    i \omega_c \left\langle J_{\bm k}(\alpha) \left[
    \frac{\gamma^2 + {\bm k}^2}{(\gamma - i {\bm k}\cdot{\bm v})^3}
   - \frac{\gamma}{\left(\gamma - i {\bm k} \cdot{\bm v}\right)^2}
     \right]                                          
     \right\rangle\ , 
\end{align}
is harder to evaluate, as it depends upon the distribution of
the particles over angles.  We shall perform the analysis
for the diffuse scattering contribution for which
the source term is proportional to~$\sin\alpha$
for~$0 < \alpha < \pi$:
\begin{align}
  J^\mathrm{(diff)}_{\bm k}(\alpha) = f_s \sin\alpha 
\end{align}
It is convenient to employ
the master integral here and in what follows:
\begin{align}
  F_m(k, q) = \int\limits_{0}^{\pi} \frac{1}{\gamma - i {\bm k} \cdot{\bm v}}
  \frac{d\alpha}{2\pi}
  = \frac{1}{2\sqrt{k^2 + q^2 + \gamma^2}}
  \left[1 + \frac{2i}{\pi} \log\frac{q + \sqrt{q^2 + k^2 + \gamma^2}}{
      \sqrt{k^2 + \gamma^2}}\right]\ . 
\end{align}
(The master integral can be calculated e.g. via the substitution
$u = \tan \alpha/2$.)
For the contribution to the density, we write
\begin{align}
  \delta\dct{\rho}^\mathrm{(diff)}(k, q) = i \omega_c f_s F_{s, \rho}({\bm k}) 
  =  i\omega_c
  f_s \int\limits_{0}^{\pi} \frac{\sin\alpha ({\bm k}\times{\bm v})}{
  \left(\gamma - i {\bm k} \cdot {\bm v}\right)^3} \frac{d\alpha}{2\pi} \ . 
\end{align}
One may notice that the expression here can be integrated by parts,
which brings it to the form
\begin{align}
  F_{s, \rho}(k, q) = \frac{i}{2} \int\frac{\cos\alpha}{
                    \left(\gamma - i {\bm k} \cdot {\bm v}\right)^2}
  \frac{d\alpha}{2\pi} \ , 
\end{align}
which can be expressed as a derivative of the master integral
with respect to~$k$:
\begin{align}
  F_{s, \rho}(k, q) = -\frac{1}{2} \frac{\partial F_m}{\partial k}
  = \frac{k}{4 (k^2 + q^2 + \gamma^2)^{3/2}}
  \left[1 + \frac{2i}{\pi} \log(\ldots)
   + \frac{2i}{\pi} \frac{q \sqrt{k^2 + q^2 + \gamma^2}}{k^2 + \gamma^2}
  \right]\ . 
\end{align}
Similarly, we find the contribution to the vorticity,
$\delta\dct{\Omega}({\bm k}) = i \omega_c f_s F_{s, \Omega}(k, q)$
with
\begin{align}
  F_{s, \Omega}(k, q) &= \int\limits_{0}^{\pi}
  \sin\alpha \left[ \frac{\gamma^2 + {\bm k}^2}{
          \left(\gamma - i {\bm k}\cdot{\bm v}\right)^3}
  - \frac{\gamma}{ \left(\gamma - i {\bm k}\cdot{\bm v}\right)^2}
  \right] \frac{d\alpha}{2\pi} \\
  &= \frac{i(\gamma^2 + k^2 + q^2)}{2}
  \frac{\partial^2 F_m}{\partial\gamma \partial q}
  + i \gamma \frac{\partial F_m}{\partial q}\ . 
\end{align}
An explicit calculation then yields
\begin{align}
  F_{s, \Omega}(k, q) =
 \frac{i \gamma q}{4(k^2 + q^2 + \gamma^2)^{3/2}}
  \left[1 + \frac{2i}{\pi} \log \frac{q + \sqrt{k^2 + q^2 + \gamma^2}}{
                          \sqrt{k^2 + \gamma^2}}
  + \frac{2i}{\pi} \frac{q\sqrt{k^2 + q^2 + \gamma^2}}{k^2 + \gamma^2}
  \right]\ . 
\end{align}
One may also note that
\begin{align}
  k F_{s, \Omega}(k, q) = i \gamma q F_{s, \rho}(k, q)
\end{align}


An isotropic injector defined by the source term
\begin{align}
  J_{\bm k}(\alpha) = 2I_0 \Theta(\alpha)
\end{align}
is treated similarly.
Its contribution to the density is given by
$\delta\dct{\rho}^{(I)} = i \omega_c I_0 F_{I, \rho}(k, q)$, with
\begin{align}
  F_{I, \rho}(k, q) = \int\limits_{0}^{\pi} \frac{({\bm k}\times{\bm v})}{
     \left(\gamma - i {\bm k} \cdot{\bm v}\right)^3} \frac{d\alpha}{\pi}\ , 
\end{align}
where the integrand is in fact proportional to the derivative
of~$(\gamma - i {\bm k}\cdot{\bm v})^{-2}$. Thus one finds
a $q$-independent contribution:
\begin{align}
  F_{I, \rho}(k, q) = \frac{2k\gamma}{\pi (k^2 + \gamma^2)^2}\ . 
\end{align}
The contribution to the vorticity
$\delta\dct{\Omega}^{(I)} = i \omega_c I_0 F_{I, \Omega}(k, q)$ with
\begin{align}
  F_{I, \Omega}(k, q) = \int\limits_{0}^{\pi} \left[
  \frac{\gamma^2 + k^2 + q^2}{\left(\gamma - i {\bm k}\cdot{\bm v}\right)^3}
  - \frac{\gamma}{\left(\gamma - i {\bm k}\cdot{\bm v}\right)^2}
  \right]
  \frac{d\alpha}{2\pi}\ , 
\end{align}
can again be expressed in terms of the derivatives of the master integral:
\begin{align}
  F_{I, \Omega}(k, q) = \left(\gamma^2 + k^2 + q^2 \right)
  \frac{\partial^2 F_m}{\partial \gamma^2}
  + 2 \gamma \frac{\partial F_m}{\partial\gamma}\ . 
\end{align}
Again, a straightforward calculation yields
\begin{align}
  F_{I, \Omega}(k, q) = -\frac{1}{2 \left(k^2 + q^2 + \gamma^2\right)}
  \left[
  \frac{k^2 + q^2}{\sqrt{k^2 + q^2 + \gamma^2}}
  \left(
  1 + \frac{2i}{\pi} \log\frac{q + \sqrt{k^2 + q^2 + \gamma^2}}{
                           \sqrt{k^2 + \gamma^2}}
  \right) \right. \\ \nonumber
  - \left.\frac{2iq}{\pi} \frac{(q^2 + k^2)(\gamma^2 - k^2) + 2 \gamma^4}{
  (k^2 + \gamma^2)^2}
  \right]\ . 
\end{align}

These equations may look rather cumbersome, but they all share the same
property: they define functions that are complex-analytic in the upper
half-plane. (Indeed, all the poles in the denominator are at~$\Im q > 0$.)
For the diffuse scattering contribution,
these expressions may be viewed as a solution of the
Riemann-Hilbert problem for the respective integrals
extended to the whole circle:
\begin{align}
%  F_{I, \omega}(k, q) &= \left[\frac{-k^2 - q^2}{
%  2\left(k^2 + q^2 + \gamma^2\right)^{3/2}}\right]^{+}\ , \\
  F_{s, \rho}(k, q)
  = \left[\frac{k}{2(k^2 + q^2 + \gamma^2)^{3/2}}\right]^{+}\ , 
  \qquad
  F_{s, \Omega}(k, q) &= \left[\frac{i \gamma q}{
          2\left(k^2 + q^2 + \gamma^2\right)^{3/2}}\right]^{+}\ . 
\end{align}
For an isotropic injector, however, the structure of the
density and vorticity contributions is more complicated.
It is in fact instructive to compare them to the respective expressions
for an injector that emits the current~$2I_0$ isotropically both into the
upper and lower semispaces: ${\tilde J}_{\bm k}(\alpha) = 2I_0$.
Computing the respective integrals, one finds
\begin{align}
  {\tilde F}_{I, \rho}(k, q) = 0 \ , \qquad
  {\tilde F}_{I, \Omega}(k, q)
  = - \frac{k^2 + q^2}{(k^2 + q^2 + \gamma^2)^{3/2}}\ . 
\end{align}
One may then solve the respective Riemann-Hilbert problem,
decomposing~${\tilde F}_{I, \Omega}(k, q)$
into the contributions analytic in the upper and lower half-planes,
respectively. This yields
\begin{align}
  \fplus{\tilde{F}}_{I, \Omega}(k, q) = F_{I, \Omega}(k, q)
  - \frac{i \gamma^2 q}{(k^2 + \gamma^2)^2}\ . 
\end{align}
We see that both~$F_{I, \rho}(k, q)$ and~$F_{I, \Omega}(k, q)$
involve the terms that do not involve the singularities
at~$q = \pm i |k|$ or~$q = \pm i\sqrt{q^2 + k^2 + \gamma^2}$,
which indicates that they are not associated with propagation
of particles in the bulk. The unwanted terms are due to
the discontinuity of the source at~$\alpha = 0$ and~$\alpha = \pi$.
These terms in fact have to be discarded. 


Extra terms:
$$
\frac{1}{(\gamma + i k)^2}
- \frac{1}{(\gamma + i k)^2}
$$
and
$$
\frac{q}{(\gamma + i k)^2}
+ \frac{q}{(\gamma + i k)^2}
$$
are due to the delta-like contributions ... 


\textbf{Coefficient not checked!}

\begin{thebibliography}{99}
\bibitem{bib:momentum-relaxation} Ref to collision integral
\bibitem{bib:wiener-hopf} Wiener-Hopf (Noble?)
\bibitem{bib:Reuter-Sondheimer} Reuter-Sondheimer
\bibitem{bib:Shytov-et-al} Shytov et al
\bibitem{bib:Levitov-Falkovich} Levitov-Falkovich
\bibitem{bib:Ilani} Scanning gate measurements: Ilani?
\bibitem{bib:Bandurin} Vicinity resistance measurements.
\bibitem{bib:Ensslin} Ensslin, scanning gate measurements
\bibitem{bib:Measuring-Psi} Measuring psi: Levitov-Falkovich theorem,
  Ilani's measurements
\bibitem{bib:Gradstein} Gradstein and Ryzhik, the integral $log(x)/(x^2 + a^2)$
\end{thebibliography}


\end{document}
